 \documentclass[fontset=founder]{ctexbook}
%\documentclass[fontset=founder]{elegantbook}

    \author{佚名}
    \title{魔幻2022}
    \pagestyle{headings}
%     \ctexset{section={
%             name={第,节},
%             number=\arabic{section}
%         }
%     }

%% 画图
%\usepackage{tikz}

%% 设置页边距
\usepackage[left=2.6cm, right=2.6cm]{geometry}

%% 设置页眉页脚
\usepackage{fancyhdr}
\pagestyle{headings}
\rhead{\chaptername \thesection - \thepage}
\chead{\date}
\rhead{\chaptername \thesection - \thepage}
\lfoot{}
\cfoot{plain}
\rfoot{}
\renewcommand{\headrulewidth}{0.4pt}
\renewcommand{\headwidth}{\textwidth}
\renewcommand{\footrulewidth}{0pt}

% 使用超链接
 \usepackage[
     colorlinks=true,
     linkcolor=red,
     filecolor=red,      
     urlcolor=blue,
     citecolor=cyan]{hyperref}
    
%% 使用语法高亮
\usepackage{listings}
\usepackage{xcolor}
% 定义可能使用到的颜色
\definecolor{CPPLight}  {HTML} {686868}
\definecolor{CPPSteel}  {HTML} {888888}
\definecolor{CPPDark}   {HTML} {262626}
\definecolor{CPPBlue}   {HTML} {4172A3}
\definecolor{CPPGreen}  {HTML} {487818}
\definecolor{CPPBrown}  {HTML} {A07040}
\definecolor{CPPRed}    {HTML} {AD4D3A}
\definecolor{CPPViolet} {HTML} {7040A0}
\definecolor{CPPGray}  {HTML} {B8B8B8}
\lstset{
    columns=fixed,       
    numbers=left,                                        % 在左侧显示行号
    frame=none,                                          % 不显示背景边框
    backgroundcolor=\color[RGB]{245,245,244},            % 设定背景颜色
    keywordstyle=\color[RGB]{40,40,255},                 % 设定关键字颜色
    numberstyle=\footnotesize\color{darkgray},           % 设定行号格式
    commentstyle=\it\color[RGB]{0,96,96},                % 设置代码注释的格式
    stringstyle=\rmfamily\slshape\color[RGB]{128,0,0},   % 设置字符串格式
    showstringspaces=false,                              % 不显示字符串中的空格
    language=c++,                                        % 设置语言
    morekeywords={alignas,continute,friend,register,true,alignof,decltype,goto,
    reinterpret_cast,try,asm,defult,if,return,typedef,auto,delete,inline,short,
    typeid,bool,do,int,signed,typename,break,double,long,sizeof,union,case,
    dynamic_cast,mutable,static,unsigned,catch,else,namespace,static_assert,using,
    char,enum,new,static_cast,virtual,char16_t,char32_t,explict,noexcept,struct,
    void,export,nullptr,switch,volatile,class,extern,operator,template,wchar_t,
    const,false,private,this,while,constexpr,float,protected,thread_local,
    const_cast,for,public,throw,std},
    emph={map,set,multimap,multiset,unordered_map,unordered_set,
    unordered_multiset,unordered_multimap,vector,string,list,deque,
    array,stack,forwared_list,iostream,memory,shared_ptr,unique_ptr,
    random,bitset,ostream,istream,cout,cin,endl,move,default_random_engine,
    uniform_int_distribution,iterator,algorithm,functional,bing,numeric,},
    emphstyle=\color{CPPViolet}, 
}

\begin{document}
 
\maketitle% show title

\tableofcontents


\chapter*{序}
一些博客文章

%%%%%%%%%%%%%%%%
\chapter{道德与法律}
\section{携子跳楼}
胡适有首小诗,不必是儿子

%%%%%%%%%%%%%%%%
\chapter{网络人物}

\section{大力出奇迹}
东北大力哥,喝止咳糖浆上瘾,劫持ATM机取款路人失败,被捕后跟记者胡言乱语,在电视台播出,B站名噪一时。

\section{打工是不可打工能的}
广西南宁周某,在偷电瓶被捕后,

我们要悄悄的打工,然后惊艳所有人。

没有困难的工作,只有勇敢的打工人。

\section{法外狂徒张三}
近两年,厚大法考的罗翔老师,在网络上比较火,人送外号“法外狂徒张三”。罗翔走红的原因,在于中国法制尚不健全。三四线城市,就业少,出路不多,普遍内卷,互相倾轧,零和游戏,办事需要托关系、走后门,看脸色,弱势市场主体无法参与公平竞争,权利寻租,导致社会资源利用效率不高。

\section{雷电法王,杨永信}
布鲁诺,被作为异端烧死。网瘾者,受到杨永信的电击治疗。

想起《发条橙》中,犯罪分子被捕入狱后,接受“厌悲疗法”的条件反射治疗。厌悲疗法不仅剥夺了他向恶的自由意志,也剥夺了他自卫的本能,他成了“发条橙”,一个没有自由意志的机械装置。

很多家长教育孩子时,都要想一下,自己认为的就一定是对的吗?权威是否是真的权威?爱因斯坦的相对论,对牛顿的经典力学,意味着什么?对与网络这种新事物,先进所有网民都沉迷的时候,那时的网瘾少年是否是在积极拥抱一个新时代呢?

\section{三和大神}

\section{消失的杀马特}
也是一个纪录片,记录了杀马特贵族的历史。

\section{闪电五连鞭}
马保国,B站捧出的神人。群众的娱乐,就是这么低级,这么无味。


\section{我的滑板鞋}
约翰逊.庞麦郎

\section{广州十年爱情故事 - 雨神}

\section{好想长出叶绿体}
\href{https://www.bilibili.com/video/av415856325/}{B站某UP民谣:好想长出叶绿体}。
我多想像大树一样

\section{郑钱花 - 川子}
可是,我的宝贝你知道么?
现在的钱有多么难挣啊?
养一个孩子他妈不容易呀!
计划生育还有必要吗?!
伟大的祖国她超有钱啊!
四万个亿跟我有蛋关系呢?
骄傲的GDP它噌噌的涨啊!
能给我换来几包尿不湿吗?!

%%%%%%%%%%%%%%%%%%
\chapter{民族与国家}

\section{新文化运动未解决的问题}
日军侵华战争,打断了新文化运动,国民在颠沛流离中,只能求生存,无力思建设。

\section{民族国家}
想象的共同体, 成为政治上的国家.
黑人身强力壮, 但战争能力几乎为零, 现代社会战争已经是以武器装备为基础的组织能力比拼, 个体在战争中起得作用极小, 不会主要依靠个人英雄人物了
\section{如何向日本学习}
近代日本一直是东亚的优等生,很像不断吸收先进文明来完善自身的古罗马。日本又像大英帝国一样,发源于海岛一隅,侵略全亚洲。

自二战胜利后国人可否有深刻反思,还是被狂热的历史仇恨蒙蔽了眼睛,以至于无法正视自己的缺点?
\section{中美差距}

\section{家庭暴力与中国女权}
《家庭、私有制和国家的起源》

\section{当种族歧视批上政治正确的外衣}
“黑命贵”导致“零元购”,也不知黑命是否还贵。

黑命贵,那是比亚裔来说,当然贵。虽然很多中国人移民美国硅谷,成功实现了“美国梦”,但他们做的大部分都是STEM相关工作,几乎没有体育、政治、艺术方面的上层精英人士,而所谓STEM,都是美国中产阶级不想做的无聊工作,所以丢给新兴国家里的优秀移民来做。当然能进入STEM行业工作,比这些新移民的母国来说,生活水平确实提升了。总体看,美国亚裔处于社会底层,比美国黑人还要底层。

美国白人给亚裔贴上“模范少数族裔(Model Minority)”标签,等于捧杀。不切实际的恭维,而没有实在的利益,为了维持这个模范形象,亚裔不敢像黑人那样到处惹事争取权益,只能忍气吞声,蜷缩在中餐馆、洗衣店、美甲店、按摩店、蔬菜大棚、超市收银。

比如,斯坦福的李飞飞教授,就是开洗衣店来补贴科研经费。

又比如,有两个亚裔up主“双琴侠(TwoViolinSet)”,故意玩梗“Lingling”,意思是,亚裔(特指中国人细分)在小提琴领域取得成功,就是靠一天40小时的残酷训练,而同样成功情况在白人身上,就是自然而然,就是gift,天赋,是上天赐予的。这种行为,无非暗示说,中国人智商低,必须靠超强度的训练才能维持高水准表演。打个不太恰当的比喻,像是,已经被驯化了的奴隶,歧视未被驯化的奴隶?这种梗,是潜在的,很多人都把它当初一种恭维,其实质是一种隐含的贬低。

所以很多白人喜欢不吝啬对移民的赞美,实际他们内心是很不在意这些移民的成功的,无论这些新移民付出多少努力,永远是you,而不是we。而第二代移民,很少能复制其父辈的成功,毕竟,第一代移民是其母国里最优秀的一部分群体,搬家穷三年,在新国家日子不会过太好,对下一代的支持相对不如在母国环境支持更多。

亚裔作为美国的干儿子,温顺老实,处处受气;黑人作为美国的亲儿子,到处滋事,不好意思去打,处处维持政治正确。

普通亚裔移民美国,不容易。如果是绝世美女,或青年身怀绝技,那么在美国还是过得很舒服的。

\section{绝命毒师老白的悲哀}
老白的悲哀在与没人理解他,别人都把他当loser。
老白这种典型理工宅男身上,好像看到,冉阿让的影子。
老白结局与毒贩同归于尽,结束了这个罪恶的世界,也算一种救赎吧。
大家喜欢这部剧,恐怕都看到了自己面对生活无奈时,还对自己有所期望。

\section{中国人以“吃货”为荣?}
这是由于中国社会不够多元造成的,从温饱水平出来的人们,还停滞在80年代物资匮乏的旧思维里,以吃为乐,以吃为荣。

西方国家,各种运动、歌剧、音乐、绘画、社区义工,社会生活非常多元,社会贫富差距比较小,人们有更多闲暇,关注自身生活,而不是靠短平快的猛吃获得存在感。

另外,某些中国饭菜,味道确实也不错。但是中餐有个致命缺点,无法标准化、国际化。那些饭菜在中国还行,到了国际上,除了价格优势,毫无竞争力。

川菜开遍全国,说到底还是西南人口向北上广一线城市。这种廉价的红油火锅,以低成本优势,重口味掩盖了食材的不佳,扩散到全国。

\section{怎么吃火锅?}
为什么有些火锅帖子总有四川人说蘸麻酱是异端? - 屁师的回答 - 知乎

https://www.zhihu.com/question/34590519/answer/1288712298

\section{迷失的苏联}

\section{《饥饿的盛世》}
张宏杰写了《饥饿的盛世》。

“盛世无饥馁,何须耕织忙”,红楼梦里,这句话格外刺耳,出自林黛玉的《杏帘在望》。大观园虽好,刘姥姥实则饥饿不堪。只有曹雪芹这种,从统治集团巅峰迅速跌落的公子哥,才能认识到封建社会,剥削底层劳动人民的残酷。


\section{望谁成龙?}
望子成龙还是望父成龙?

一直认为需要英雄的国家是悲哀的民族。需要龙的家庭,是无聊乃至失败的家庭。家族的兴衰,在于每个人都努力一点,所谓日拱一卒,功不唐捐。而不是靠救世主,靠英雄人物拯救自己。

真正的强者,身体力行,谁也不指望谁。


\chapter{教育}

\section{教什么}
\section{怎么教}
\section{如何解决钱学森之问}
根本上还是人均GDP不高,教育部门作为成本消耗方,对拉动GDP无法立竿见影。

美国有无数优秀高校,中国才34所211/985大学,即便新推出双一流计划,仍然是换汤不换药,旧酒装新瓶,存量博弈,而且落后高校几乎无法实现反超。

\section{为什么贫穷纪录片:教育出路}
这是一款纪录片的标题,记录了中国农村孩子的高考之路。

片中,看到女孩的母亲没了一只手,还在努力地搬砖。

\section{衡水中学后遗症}
衡水中学,挑战北方一切县级中学。

这是一个不需要超级中学的时代。衡水中学,河北教育的伤疤、畸形、怪胎、毒瘤、精华。

衡水中学,永远滴神。

\section{大山里的女校校长,张桂梅}
希望这个时代,不再出现英雄,不再存在模范,不再有最美的某某某,希望我们平庸就幸福,希望人生不必波澜壮阔,我只想做个快乐的废宅,每天上班摸鱼,下班打游戏。

职业教育孱弱,得不到认同,所以张桂梅只能组建普通高中。
 
\end{document}
